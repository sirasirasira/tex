\section{Depth-first Search with Branch and Bound}
\label{sec:previous}
The state of the art methods \cite{Yan:2008, Saigo:2009, Shirakawa:2018} 
searching for a discriminative subgraph pattern is typically based on the depth-first search, 
as shown in Figure \ref{fig:search} (a), with a branch and bound technique.
\cite{Shirakawa:2018} searches the best subgraph pattern based on $DScore$. 
This method calculates the lower bound of $DScore$ obtained by child nodes of search space, 
and if this lower bound does not reach the provisional solution the child is pruned.
\begin{comment}
\begin{theorem}
  \label{thm:bound_C}
  Given $\D_1(g)$ and $\D_0(g)$, for any subgraph $g' \sqsupseteq g$,
  \begin{multline}
    \label{eq:bound_C}
    CScore(g') \leq 
    \max \Big[ 2 \sum_{\{n | y_n=+1, \{G_{n}, y_{n}\} \in D_1(g)\}} y_n - \sum_{n=1}^{N}y_n, 
	-2 \sum_{\{n | y_n=-1, \{G_{n}, y_{n}\} \in D_1(g)\}} y_n + \sum_{n=1}^{N}y_n \Big]
  \end{multline}
\end{theorem}
\end{comment}

\begin{theorem}
  \label{thm:bound_R}
  Given $\D_1(g)$ and $\D_0(g)$, for any subgraph $g' \sqsupseteq g$,
  \begin{multline}
    \label{eq:bound_R}
    DScore(g') \geq 
    \mymin{(\diamond,k)} \Big[ \tss(\D_1(g) \setminus S_{\diamond, k}) + \tss(\D_0(g) \cup S_{\diamond, k}) \Big]
  \end{multline}
  where $ (\diamond, k) \in \{ \leq, > \} \times \{ 2, \dots, |\D_1(g) - 1| \} $,
  and $S_{\diamond, k} \subset \D_1(g)$, such that $S_{\leq, k}$ is
  a set of $k$ pair $(G_i, r_i)$ selected from $\D_1(g)$ in descending order of residual error $r_i$,
  and $S_{>, k}$ is that in increasing order.
  Note that $\setminus$, $\cup$ are set difference and set union respectively.
\end{theorem}

\begin{proof}
  Given $\D_1(g)$ and $\D_0(g)$,
  \small
  \begin{align}
    bound(g) &= \mymin{g'} \bigl[ \tss(\D_1(g')) + \tss(\D_0(g'))  \bigr] \notag \\
    \label{eq:boundSubset}
    &= \mymin{S \subset \D_1(g)} \bigl[ \tss(\D_1(g) \setminus S) + \tss(\D_0(g) \cup S)  \bigr] \\
    \label{eq:linearBound}
    &= \mymin{(\diamond,k)} \Big[ \tss(\D_1(g) \setminus S_{\diamond,k}) + \tss(\D_0(g) \cup S_{\diamond,k}) \Big]
  \end{align}\normalsize
  where $ (\diamond, k) \in \{ \leq, > \} \times \{ 2, \dots, |\D_1(g) - 1| \} $.
  From the anti-monotone property \eqref{eq:propSubgraph}, we have
  $\D_1(g') \subseteq \D_1(g)$ for $g' \sqsupseteq g$
  for the training set $\D$ from which the equation \eqref{eq:boundSubset}
  directly follows. Thus, we show \eqref{eq:linearBound} in
  detail. For simplicity,
  let $A =,\{ a_1, \dots, a_n \mid a_i \in \mathbb{R} \}$ denote $\D_1(g)$,
  and $B = \{ b_1, \dots, b_m \mid b_i \in \mathbb{R} \}$ denote $\D_0(g)$.
  Then, the goal of \eqref{eq:boundSubset} is to minimize the total sum
  of squares $\tss(A \setminus S) + \tss(B \cup S)$ by tweaking
  $S = \{ s_1, \dots, s_k \} \subset A$.
  The key fact is that $\tss(A \setminus S) + \tss(B \cup S)$ can
  be regarded as a quadratic equation of $\sum_{i=1}^k s_i$ when the size
  of $S$ is fixed to $k$. More precisely,
  \begingroup
  \allowdisplaybreaks
	\begin{align*}
	  &\tss(A) 
	  = \mysum{n} \Big(a_i - \frac{\mysum{n}a_i}{|A|}\Big)^2
	  = \mysum{n} a_i^2 - \frac{\big(\mysum{n} a_i\big)^2}{|A|} \\
	  &\tss(A) + \tss(B) 
	  = \mysum{n} a_i^2 + \mysum{m} b_i^2 - \frac{\big(\mysum{n} a_i\big)^2}{|A|} - \frac{\big(\mysum{m} b_i\big)^2}{|B|} \\
	  &\tss(A \backslash S) + \tss(B \cup S) \\
	  &= \mysum{n} a_i^2 + \mysum{m} b_i^2 - \frac{\big(\mysum{n} a_i - \mysum{k} s_i\big)^2}{|A \backslash S|} - \frac{\big(\mysum{m} b_i - \mysum{k} s_i\big)^2}{|B \cup S|} \\
	  &= \mysum{n} a_i^2 + \mysum{m} b_i^2 - \frac{\big(\mysum{n} a_i - \mysum{k} s_i\big)^2}{n - k} - \frac{\big(\mysum{m} b_i - \mysum{k} s_i\big)^2}{m + k} \\
	\end{align*}
      \endgroup
      Therefore, $\tss(A \setminus S) + \tss(B \cup S)$ is minimized
      when $\sum_{i=1}^k s_i$ is maximized or minimized. In other words,
      \eqref{eq:boundSubset} becomes minimum when the mean of $S \subset
      \D_1(g)$ is maximized or minimized.
	  \qed
\end{proof}
This computational cost is $O(|\D_1(g)|)$, i.e. it takes linear time for the frequency of the subgraph.
