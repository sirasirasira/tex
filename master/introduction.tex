\section{Introduction}
Graphs are fundamental data structures for representing combinatorial objects 
such as chemical compounds, networks and others.
However, precisely because of their combinatorial nature, 
it is usually difficult to understand the underlying trends in large datasets of graphs.
In addition, due to the rapid increase in the amount of data in recent years, 
supervised learning in which the inputs are graphs of arbitrary size and shape has attracted attention
in the fields of computer vision \cite{Harchaoui:2007, Nowozin:2007, Barra:2013, Bai:2014a},
chemoinformatics \cite{Kashima:2003, Tsuda:2007,Saigo:2008a, Saigo:2009,Mahe:2009, 
Vishwanathan:2010, Shrvashidze:2011, Takigawa:2013, takigawa:2017},
bioinformatics\cite{Borgwardt:2005, Karklin:2005, Takigawa:2011b} 
and computational chemistry\cite{Kearnes2016, gilmer:2017}.
and natural language processing\cite{Kudo:2005}. 

For this problem, the most fundamental feature is subgraph indicators 
because many reactions and events are attributed to substructures.
However, the number of all possible subgraph patterns is intractably large 
due to the combinatorial explosion, it is difficult to make feature from dataset.
Therefore, it is necessary to make feature by restricting the subgraph based on some criterion.
Frequent subgraph mining \cite{Yan:2002, Nijssen:2004} is one of these methods and is often used.
However, since this method only considers the frequency, 
it is possible to overlook the important features for learning.
On the other hand, discriminative pattern mining \cite{Fan:2008, Saigo:2009, Shirakawa:2018} 
extracting features important for learning using model-based discriminative criteria has been studied.
Based on these methods, the present paper considers efficient discriminative subgraph pattern search methods.

\subsection{Related Work}
\label{sec:relatedwork}
Related works \cite{Saigo:2009, Shirakawa:2018} search the discriminative subgraph pattern
using depth first policy and branch and bound method.
While these methods perform efficient searches by designing tricky pruning, 
search policy of depth first is naive and 
they still suffer from computational costs for some domain or large datasets.

\subsection{Our Approach}
\label{sec:ourapproach}
To overcome the above difficulty, 
we consider the more efficient search algorithms
using Best First Search and Monte Carlo tree search(MCTS).
These methods are known to be effective in some domains, 
and in this paper we apply to subgraph search domain.
