\section{Introduction}
Graphs are fundamental data structures for representing combinatorial objects 
such as chemical compounds, networks and others.
However, precisely because of their combinatorial nature, 
it is usually difficult to understand the underlying trends in large datasets of graphs.
The rapid increase in data in recent years also includes data
represented as graphs, and thus supervised learning in which the inputs are graphs of
arbitrary size and shape has gained considerable attention
in the fields of computer vision \cite{Harchaoui:2007, Nowozin:2007, Barra:2013, Bai:2014a},
chemoinformatics \cite{Kashima:2003, Tsuda:2007,Saigo:2008a, Saigo:2009,Mahe:2009, 
Vishwanathan:2010, Shrvashidze:2011, Takigawa:2013, takigawa:2017},
bioinformatics\cite{Borgwardt:2005, Karklin:2005, Takigawa:2011b} 
and computational chemistry\cite{Kearnes2016, gilmer:2017} 
and natural language processing\cite{Kudo:2005}. 

Some of these problems have graph descriptors prepared in advance, 
such as molecular dataset\cite{james:2004, durant:2002}, 
but others have no direct descriptors available.
For these problems, the most fundamental features are subgraph indicators, 
i.e., if a subgraph is embedded in the given graph or not,
because many reactions and events are attributed to those substructures.
However, the number of all possible subgraph patterns is intractably large 
due to the combinatorial explosion, 
so that it is difficult to enumerate all subgraphs in a realistic time for a given dataset.
Therefore, it is necessary to restrict the subgraph candidate on the basis of some criterion.
Frequent subgraph mining \cite{Yan:2002, Nijssen:2004} is one method in this approach.
However, since it chooses subgraphs acording to the frequencies, 
it is possible to overlook the importance of subgraphs.
On the other hand, discriminative pattern mining techniques \cite{Fan:2008, Saigo:2009, Shirakawa:2018} 
choose subgraphs according to a model-based discriminative criteria, 
and these not likely to overlook important patterns for learning.
Because of these merits, the present paper also adopts this approach.

\subsection{Related Work}
%TODO
\label{sec:relatedwork}
Related works \cite{Saigo:2009, Shirakawa:2018} search the discriminative subgraph pattern
using depth first policy with branch and bound method.
While these methods perform efficient searches by designing tricky pruning, 
search policy of depth first is naive and 
they still suffer from computational costs for some domain or large datasets.

\subsection{Our Approach}
\label{sec:ourapproach}
To overcome the above difficulty, 
we propose the more efficient two algorithms
using Best First Search and Monte Carlo tree search (MCTS).
The efficiency of these building-block algorithms are already known in some domains, 
so that in this paper we apply them to our graph classification/regression problem.
