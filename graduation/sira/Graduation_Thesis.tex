\documentclass[12pt,fleqn]{jsotsuron}
\setlength{\mathindent}{2zw}
\usepackage[dvipdfmx]{graphicx}
\usepackage{makeidx}
\usepackage{amsmath,amssymb}
\usepackage{amsthm}
\usepackage{amsfonts}
%\usepackage{algorithmic}
\usepackage{algorithm}
\usepackage{algpseudocode}
\usepackage{url}
\usepackage{here}
\usepackage[T1]{fontenc}
\usepackage{fancyhdr}
\usepackage{listings}
\usepackage{multirow}
\usepackage{comment}
\usepackage{enumerate}
\usepackage{bm}
\usepackage{latexsym}
\usepackage{color}
\usepackage{caption}
\usepackage{subcaption}
\usepackage{fleqn}
\usepackage{nccmath}
\usepackage{setspace}

\DeclareMathOperator*{\argmin}{arg\,min}
\DeclareMathOperator*{\argmax}{arg\,max}

%\renewcommand{\algorithmicrequire}{\textbf{Input:}}
%\renewcommand{\algorithmicensure}{\textbf{Output:}}
\renewcommand{\proofname}{証明}
\newtheorem{Theorem}{定理}
\newtheorem{Definition}{定義}
\newtheorem{Proof}{証明}[section]
\renewcommand{\proofname}{証明}
\renewcommand{\theProof}{}
\newtheorem{Lemma}{命題}[section]
\def\vector#1{\mbox{\boldmath $#1$}}
\bibliographystyle{jplain}

\title{グラフ分類における部分グラフ特徴集合の確率的探索\\
``Probabilistic search of subgraph feature sets in graph classification''}
\author{白川 稜}
\date{平成30年2月}
\makeindex

\begin{document}
\maketitle
\tableofcontents
\newpage

\abstract{
	グラフ分類問題では部分グラフの有無を特徴量とすることが多い.
	一般に入力グラフにおける部分グラフ数は膨大であり,全列挙は現実的ではない.
	これに対して既存手法gBoostでは,特徴探索と分類器の学習を同時に行うことで,
	部分グラフを全列挙せずに全部分グラフを用いた学習を可能にする.
	しかし,gBoostではある目的関数を最大にする部分グラフを厳密に探索するため,
	特徴探索の面で多大なコストを要する.
	ここで,部分グラフ探索空間には似た部分グラフが多数存在しており,厳密に探索を行う必要がない場合もあるとかんがえられる.
	したがって本研究では,グラフ分類問題に対して,確率的に探索空間を削減し,部分グラフ特徴探索を効率化する手法を提案する.
}

\chapter{はじめに}
\section{背景}
グラフは,自然言語処理\cite{Adaboost},RNA二次構造\cite{DBLP:conf/psb/KarklinMH05},低分子化合物の構造式\cite{Takigawa201350}などの知識処理に幅広く用いられる重要なデータ構造である.
近年こうした科学分野のグラフデータが公共データベースに蓄積,整備されるようになり,有効な利活用が課題となっている.
特に,グラフデータからの教師付き学習は,生命科学や物質科学における構造活性相関や構造物性相関の定量的モデルとして機械学習分野で研究されており,より高精度で効率的な手法が求められている.

グラフデータからの教師付き学習では,特徴量として部分グラフの有無(部分グラフ指示子)を用いることが多い.
この場合,検査する部分グラフ特徴の選択が学習モデルの予測精度を左右する.
しかし,適切な特徴集合はデータによって異なる.
一方で,与えられたグラフデータに生起する全ての部分グラフの列挙は現実的ではない.
したがって,グラフカーネル法\cite{DBLP:journals/jmlr/ShervashidzeSLMB11}やECFP法\cite{DBLP:journals/jcisd/RogersH10}など,対象となる部分グラフのクラスを予め発見的に制限する手法や,
gBoost法\cite{gBoost}やAdaboostに基づく手法\cite{Adaboost}など,必要な部分グラフのみを効率的に探索しながら学習を行う方法が提案されている.
後者のアプローチは候補集合から必要な部分グラフのみを探索でき,広い種類のグラフデータに対する汎用性が期待できる.
一方で,必要な部分グラフを厳密に何度も探索するため,
与えられたグラフデータのグラフ数及び含まれるグラフの大きさによっては
現実的な時間でモデル構築をするのが困難になる場合もある.

\section{概要}
そこで,本研究では,部分グラフ指示子の探索を効率化するために,
確率的アルゴリズムを用いた特徴量探索手法を提案する.
確率的アルゴリズムには,強化学習やモンテカルロ木探索法\cite{MCTS}などがあり,
扱う探索空間の規模が大きいとより効果が期待できることが知られている.
したがって,本研究で扱うグラフ分類問題において,
既存手法におけるモデル構築のボトルネックである特徴量探索の効率化を行うために
確率的アルゴリズムを取り入れる.
%TODO
%%%実際にやったことをまとめてここに書く
提案手法では,探索における確率の振り方を三種類定め,
それぞれの確率を利用した三種類の確率的探索手法を考える.
各手法の評価を,実データを用いた実験により比較する.

\section{構成}
本論文では,前節の概要について,以降の章で以下の通り詳述する.
\begin{itemize}
\item 第2章では,頻出部分グラフマイニングの記法,定義について述べる.
\end{itemize}

\section{既存手法(gBoost)}
本章では、既存手法であるgBoost\cite{gBoost}について説明する。
gBoostは線形計画問題で定式化されたブースティング手法であるLPBoost\cite{lpboost}と、
部分グラフの探索アルゴリズムを組み合わせたグラフの分類、回帰モデルである。
本研究ではグラフの分類問題を扱うため、以下では、グラフの線形分類モデル、LPBoost、特徴探索アルゴリズム、と順を追って説明する。

\subsection{グラフ線形分類モデル}
まず、グラフに対する線形分類モデルについて説明する。
一般的に機械学習の線形モデルは、$n$個の特徴$x_1,x_2, \ldots ,x_n$に対して、
$f(\bm{x}) = a_1x_1+a_2x_2+ \cdots +a_nx_n+b$とし、
2値分類(+1 or -1)の場合、$f(x) \geq 0$の時に1、$f(x) \leq 0$の時に-1と予測する。
グラフ分類問題に用いる特徴は様々なものが考えられるが、
gBoostではグラフデータに含まれる部分グラフの有無を特徴として用いる。
入力として$l$個のグラフとクラスラベル$\{G_n, y_n\}^l_{n=1} (y_n \in \{+1, -1\})$が与えられた時、
入力データに少なくとも1回は出現する全ての部分グラフを$\mathcal{T}$とする。
この時グラフ$G_n$は$|\mathcal{T}|$次元のベクトル$\bm{x}_n(\bm{x} = \mathbb{I}(t \subseteq G_n), \forall t \in \mathcal{T})$として表現することができる。
$I(\cdot)$は括弧内が真なら1、偽なら0を返す関数である。
例を図1に示す。\\
この時、各特徴に対して仮説(hypotheses)と呼ばれる、弱学習器を以下で定義する。

\begin{equation}
	h(\bm{x};t,\omega) = \omega (2x_t - 1).
\end{equation}
$\omega \in \Omega = \{-1,1\}$はパラメータである。
この弱学習器を用いて、gBoostでは以下の線形分類モデルを構築する。
\begin{equation}
	f(\bm{x}) = \sum_{(t,\omega) \in \mathcal{T}\times \Omega} \alpha_{t,\omega} h(\bm{x};t,\omega).
\end{equation}
$\alpha_{t,\omega}$は各弱学習器の重みであり、$\sum_{(t,\omega) \in \mathcal{T}\times \Omega} \alpha_{t,\omega}=1,\alpha_{t,\omega} \geq 0$を満たす。


\subsection{LPBoost}
gBoostでは、
LPBoostで定式化された以下の線形計画問題を主問題として解くことで、
先ほど定義した線形モデルの重み$\alpha_\{t,\omega\}$の値を導く。

\begin{eqnarray}
	\min_{\bm{\alpha},\bm{\xi},\rho} -\rho + D \sum_{n=1}^{l} \xi_{i}\\
	s.t. \sum_{(t,\omega) \in \mathcal{T} \times \Omega} y_{n} \alpha_{t,\omega} h(\bm{x_{n}};t,\omega) + \xi_{n} \geq \rho,\\
	 \xi_{n} \geq 0,\nonumber \\
	 n = 1, \dots , l,\nonumber \\
	 \sum_{(t,\omega) \in \mathcal{T} \times \Omega} \alpha_{t, \omega} = 1, \nonumber\\
	 \alpha_{t, \omega} \geq 0. \nonumber
\end{eqnarray}

ここで、$\rho$はソフトマージン、$D=\frac{1}{\nu l}, \nu \in (0,1)$であり、$\nu$は誤分類に対するコスト制御のパラメータである。
一般に、変数$\bm{\alpha}$の数は部分グラフの総数であるため膨大であり、主問題を解くことは困難である。
したがって、主問題と等価な以下の双対問題を扱う。

\begin{eqnarray}
	\min_{\bm{\lambda},\gamma} \gamma\\
	s.t. \sum_{n=1}^{l} \lambda_{n} y_{n} h(\bm{x_{n}};t,\omega) \leq \gamma,\\
	\forall(t,\omega) \in \mathcal{T} \times \Omega, \nonumber\\
	\sum_{n=1}^{l} \lambda_{n} = 1, \nonumber\\
	0 \leq \lambda_{n} \leq D, \nonumber\\
	n = 1, \dots , l. \nonumber
\end{eqnarray}

双対問題は、主問題に比べ制約式の数は多くなるが列生成法を用いることで解を得ることが可能となる。
列生成法は、制約なしの状態から始め、現制約を最も違反する制約を追加し、
最適化問題を解くという過程を繰り返すことで解を得る手法である。
最も違反する制約を見つける問題は以下の最大化問題と等価である。

\begin{equation}
	(t^{*}, \omega^{*}) = \argmax_{t \in \mathcal{T} , \omega \in \Omega} \sum_{n=1}^{l} \lambda_{n}^{(k)} y_{n} h(\bm{x_{n}} ; t, \omega).
\end{equation}

これに対して最大化する目的関数をgainと呼び以下の式で定義する。

\begin{equation}
	g(t, \omega) = \sum_{n=1}^{l} \lambda_{n}^{(k)} y_{n} h(\bm{x_{n}} ; t, \omega).
\end{equation}

もし以下を満たすような弱学習器が存在しない場合、現在の解が双対問題の最適解となる。

\begin{equation}
	\sum_{n=1}^{l} \lambda_{n}^{(k)} y_{n} h(\bm{x_{n}} ; t, \omega) > \gamma^{(k)}.
\end{equation}

経験的に終了付近の制約追加による重みの更新は僅かであるため、
以下のように終了条件を緩和することで、計算を効率化する。

\begin{equation}
	\sum_{n=1}^{l} \lambda_{n}^{(k)} y_{n} h(\bm{x_{n}} ; t, \omega) > \gamma^{(k)} + \epsilon .
\end{equation}

双対問題の終了条件緩和により得られる主問題の解は、$V^*$を緩和なしでの最適解とすると、$V \leq V^* + \epsilon$を満たす。


\subsection{特徴探索アルゴリズム}
特徴探索は重複無しに全部分グラフを列挙した探索空間であるDFSコード木\cite{gSpan}を用いる。
DFSコード木は、全部分グラフをノードに含む他に、子が親のスーパーグラフとなるという特徴を持つ。
この探索木を深さ優先的にたどり、各ノードにおいてgainを計算することで、最適な部分グラフを探索する。
一般に、部分グラフの総数は膨大となるため全てのノードを探索するためには相当なコストを要する。
しかし、子が親のスーパーグラフとなる特徴を利用すると、
bound(子孫ノードでのgainの上限値)を計算することができるため、
安全な枝刈りを行うことが可能である。
boundの計算、枝刈り可能条件を以下に示す。

現探索部分グラフを$t$、そのスーパーグラフを$t'$とすると、
boundの値$b(t)$は以下の式で計算できる。

\begin{eqnarray}
	b(t) = \max\{2 \sum_{\{n|y_{n}=+1, t \subseteq G_{n}\}} \lambda_{n}^{(k)} - \sum_{n=1}^{l} y_{n} \lambda_{n}^{(k)},\nonumber \\
	2 \sum_{\{n|y_{n}=-1, t \subseteq G_{n}\}} \lambda_{n}^{(k)} - \sum_{n=1}^{l} y_{n} \lambda_{n}^{(k)}\}.
\end{eqnarray}

枝刈り可能条件は、現探索までのgainの最大値を$g^{*}$とすると以下となる。

\begin{equation}
	g^{*} > b(t).
\end{equation}

この枝刈りが有効に働くため、gBoostアルゴリズムは探索空間に対して効率よくモデルを構築することができる。

\section{提案手法}
既存手法のgBoostでは探索木上を深さ優先的に厳密探索する。
しかし、入力グラフ集合に対して、個々のグラフの大きさや全体数の増加に伴い、
存在しうる部分グラフの総数は急激に増加するため、厳密探索におけるコストは莫大なものである。
問題によってモデル構築が困難なものも存在する。
本稿では、厳密探索を行わず、乱数を用いて探索をランダム化する手法を提案する。

まず初めに、本提案手法で用いる$flow$の概念を説明する。\\
1.ある定数$k$を定める。\\
2.探索木の根ノードから探索を開始する。\\
3.現探索ノードにおける子ノードを列挙する。\\
4.列挙した$m$個の各子ノード$i$に対して、確率$p(i) (1 \leq i \leq m)$を定める。\\
5.定めた確率に基づいて、入力定数$k$を各子ノードに割り振る。\\
6.各子ノードにおいて、割り振られた数を入力定数とし3〜5を再帰的に行う。\\
7.探索が末端ノードに到達した場合、入力定数が1以上であっても再帰を終了する。
(ここで末端ノードとは子ノードを持たないノード、枝刈り可能なノード、指定したグラフの大きさを上回る部分グラフを表現するノードのことを指す。)

以上の探索を$k\ flow$探索と定義する。
Algorithm\ref{alg_main}に擬似コードを示す。
%アルゴリズムがここに入る
\begin{algorithm}
  \small 
\caption{探索のランダム化}\label{alg_main}
\begin{algorithmic}[1]
\Procedure{最適な部分グラフ、あるいはその共起}{}
\State グローバル変数 : $g^*,\omega^*,p^*,pc^*$
\State 最適な部分グラフ$g^*,\omega^*,p^*$とTop-k部分グラフを探索 \Comment{gBoost法の探索}
\ForAll{$p \in $ Top-k 部分グラフ}
\State project(p)
\EndFor
\EndProcedure

\Function{tmp}{$p,t$}
\If {$b(t) < g^*$}
\State \textbf{return}
\EndIf
\EndFunction
\end{algorithmic}
\end{algorithm}

%アルゴリズムの説明がここに入る

本研究では探索手順における、子ノードの選択確率$p_i$に関して、以下に示す3つの確率設定手法を提案する。

\subsection{提案手法1:一様確率設定}
提案手法1では、子ノードに対して一様に確率を設定する。
現探索ノードにおける$i$番目の子ノードの選択確率を以下に示す。

\begin{equation}
	p(i) = 1 / m.
\end{equation}

\subsection{提案手法2:支持度(support)の情報を加えた確率設定}
提案手法1は、入力グラフの特徴を利用しないナイーブな手法である。
本節では、入力グラフに対する支持度(support)の情報を利用した手法を提案する。
支持度とは、ある部分グラフが全入力グラフの内いくつのグラフで現れるかを表す値であり以下の式で定義される。

\begin{equation}
	support(x) = \sum_{n=1}^{l} \mathbb{I} (x \subseteq G_n) (x \in \mathcal{T}).
\end{equation}

探索木の性質上、支持度が大きい部分グラフのほうが小さいものに比べて
複数の入力グラフ上での拡大グラフを考えることができるため、子孫空間が広いことを期待できる。
子孫空間が狭いノードに多くの$flow$を流す場合、探索が広がらず無駄になる可能性が大きい。
したがって、本手法では支持度が大きい子ノードが選択されやすくなるよう確率を定める。
$i$番目の子ノードが表す部分グラフを$x_i$とすると、子ノードの選択確率を以下の式で示す。

\begin{equation}
	p(i) = support(x_i) / \sum_{n=1}^{m} support(x_n).
\end{equation}

\subsection{提案手法3:bound,gainの情報を加えた確率設定}
本節では、gainとboundの情報を利用した手法を提案する。
gain、boundは、本稿第2章で定義したものを用いる。
boundは子孫ノードで得ることのできるgainの上限値を表すため、
boundの大きな子ノードへの探索を行うことで、より大きなgainの値を得ることが期待できる。
また、gainの値はboundの値との探索空間上の距離を計る指標になることが期待できるため、
gainとboundの凸結合の値を用いて子の選択確率を定めることで、より効率的な探索を行う。
子ノードの選択確率を以下の式で示す。

\begin{eqnarray}
	p(i) & = & val(x_i) / \sum_{n=1}^{m} val(x_n), \\
	val(x_i) & = & \lambda gain(x_i) + ( 1 - \lambda ) bound(x_i). \nonumber 
\end{eqnarray}

$\lambda$はgainとboundの重みを表すパラメータであり、$0 \leq \lambda \leq 1$を満たす。

\subsection{探索空間の同型判定除外}
既存手法で用いられるDFSコード木はgSpan\cite{gSpan}を用いて構築される木である。
gSpanアルゴリズムでは、グラフ同型問題を解き、
同型である場合には当アルゴリズムで定義される辞書順を用いて最小のものだけを木に採用する。
当アルゴリズムにおいて辞書順最小のものは木全体に対して左側に出現しやすいため、
DFSコード木は木全体の形が左寄りになるという特徴を持つ。
ここで問題となるのが、探索木が左寄りである時、提案手法1のように一様に確率を設定する場合においても、
同階層のノードへの探索確率に大きな差が生じてしまうということである。理解を簡単にするため、以下に図を示す。

図

本研究では、gSpanアルゴリズムでのグラフ同型判定を行わずに、
辞書順最小でないノードも木に追加することで、
木全体の偏りを緩和し、以上のような探索確率の差が軽減するよう試みる。
提案手法2、3においても、同一の探索木を利用する。


\section{実験\&結果}
提案手法を用いて構築されるモデルの評価を行うために、3つの実験を行う。

\subsection{データセット}
\begin{table}
	\centering
	\scalebox{0.75}{
		\begin{tabular}{|c|r|r|r|r|r|}
			\hline
			データ名 & グラフ数 & 平均ノード数 & 平均エッジ数 & 正例 & 負例\\
			\hline \hline
			CPDB  & 684 & 25.2 & 25.6 & 341 & 343\\
			\hline
			Mutag & 188 & 26.3 & 28.1 & 125 & 63\\
			\hline
		\end{tabular}
	}
	\caption{使用したデータセット}
	\label{dataset}
\end{table}

本稿では、Saigoら\cite{gBoost}によって用いられた2つのデータセットを使用した。
表\ref{dataset}に各データセットのグラフ数、平均ノード・エッジ数、正例と負例の数を示す。

\subsection{実験1:探索の精度、効率に関する実験}
%%%%探索に関する精度(どれだけ探索出来てるか)を計る実験 < 目的関数の値で評価
実験1では、CPDBとMutagのデータに対して各提案手法がどれだけ精度よく探索できているか、
どれだけ既存手法より効率化できているかを計る実験を行う。
本稿では、主問題における目的関数(2章(5)式)の値を探索精度の指標に、
モデル構築までにgainとboundの値を計算したノードの総数を探索ノード数とし、効率化の指標に用いる。

以下では、誤分類に対するコスト制御のパラメータ$\nu$の値が0.1、0.3、0.5の3つの場合に対して実験を行う。

\subsubsection{flow - 目的関数}
$flow$の数を変化させた時の目的関数の最終値の推移を調べる。
$flow$の刻み幅は、提案手法1、2に対しては10とする。
提案手法3は、アルゴリズム上、子ノードのgainとboundを全て計算する必要があるため、
提案手法1,2と比べて探索ノード数が多くなる。したがって、$flow$の刻み幅は1とする。

\subsubsection{結果}

\subsubsection{flow - 探索ノード数}
$flow$の数を変化させた時の探索ノード数の推移を調べる。
$flow$の刻み幅は、前述したものと同じ値とする。

\subsubsection{結果}

\subsubsection{探索ノード数 - 目的関数}
各提案手法の探索ノード数に対して、目的関数の最終値の推移を調べる。

\subsubsection{結果}

\subsection{実験2:$\lambda$ に関する実験}
実験2では、提案手法3におけるgainとboundの重みパラメータ$\lambda$の値を変化させた時の目的関数の最終値、
探索ノード数の推移を調べる。
$flow$数を100に固定し、$\lambda$の値を0から1まで0.1ずつ変動させる。
これをデータに対し10 fold cross varidationを行う。 
以上の実験を実験1と同様に、$\nu$の値が0.1、0.3、0.5の場合に対して行う。

\subsection{実験3:乱数のブレに関する実験}
提案手法では、乱数を使用しているため、結果にブレが生じる。
したがって、実験3では各提案手法が乱数にどれだけ影響を受けるのか、そのブレを調べる。
$flow$数を100に固定し、乱数に10個のシード値を与えた時の目的関数の最終値、探索ノード数の平均、分散値を調べる。
以上の実験を実験1と同様に、$\nu$の値が0.1、0.3、0.5の場合に対して行う。



\chapter*{謝辞}
\addcontentsline{toc}{chapter}{謝辞}

\bibliography{ref}

\end{document}
