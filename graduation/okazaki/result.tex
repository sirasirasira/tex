
\chapter{まとめ}
\section{結論}
本研究では,頻出部分グラフマイニングのアルゴリズムであるgSpan,
さらに$\delta$-tolerance closed frequent subgraphs algorithm
を拡張し,wildcardを許容した頻出部分グラフパターンの列挙と
その飽和パターン集合・極大パターン集合を求める手法を提案した.
さらに実験から,wildcardを許容することによる頻出パターン数の増加を示し,
飽和パターン・極大パターンによるパターン集合要約の効果を確かめた.
さらに,wildcardを許容した頻出部分グラフが実際に有用なパターンで
あるかを,機械学習に対する特徴として用いることで効果検証した.
その結果,グラフデータセットによっては効果的な出力があり,
wildcardを許容に対して有効であると言える傾向実験結果を得た.
また,冗長なパターンを削減することで,精度を下げずに,
よりうまく応用できる.
%いい結果と呼べること

\section{今後の展望}
今回検証した実験に対して,さらに実験を行うことで
今回得た結果に対する考察の信憑性を確かめる必要があるだろう.
また,今回は正解率による検証を行ったが,
ここからグラフをよりうまく分類できている特徴を
見ることで,wildcardを許容した部分グラフがどれほど
選ばれているのか調べる必要がある.
さらに,通常,minsupで枝刈りをする方法は,データベース中の
すべてのグラフをうまく説明できないため,
十分にminsupを下げなければ
機械学習の特徴としてはあまり精度がでない.
minsupでは枝刈りをせず,部分グラフのエッジ数で決まるパターンの長さ
で枝刈りをする方法もあり,これに対しても
検証を行ったが,うまく説明できるような結果でなかったため,
さらに実験をし,傾向を確認したい.
