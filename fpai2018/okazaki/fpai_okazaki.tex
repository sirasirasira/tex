%%%%%%%%%%%%%%%%%%%%%%%%%%%%%%%%%%%%%%%%%%%%%%%%%%%%%%%%%%%%%%%%%%%%%%%%%%%%
\documentclass[a4j]{jarticle}
\usepackage{jsaisig}

\usepackage[T1]{fontenc}
\usepackage{times,eucal,mathptmx}
\usepackage{fancyhdr}

\usepackage{listings}
\usepackage[dvipdfmx]{graphicx}


\usepackage{multirow}
\usepackage{comment}
\usepackage{algorithm}
\usepackage{algpseudocode}
%\usepackage{algorithmic}
\usepackage{enumerate}
\usepackage{bm}
\usepackage{amsmath}
\usepackage{amssymb}
\usepackage{amsfonts}
\usepackage{latexsym}
\usepackage{color}
%\usepackage{mediabb}
\usepackage{amsthm}

%%%%%%%%%%%%%%%%%%%%%%%%%%%%%%%%%%%%%%%%%%%%%%%%%%%%%%%%%%%%%%%%%%%%%%%%%%%

\begin{document}

\title{和文タイトル}

% 英文タイトル
\etitle{English Title} 

% 著者名: 
%	・各著者を\quad(全角空白)区切りで列挙
% 	・著者名の直後に\afil{所属番号}を追加→所属番号を上付で出力(\textsuperscript{所属番号}と同じ)
% 	 複数機関へ所属している場合は番号をカンマ区切りで列挙(下記著者2参照)
%  ・Corresponding Authorについては所属の後に\thanksを続け,連絡先を記入
%	・英文著者はカンマ区切りで列挙

\author{著者1氏名\afil{1}%
	\thanks{連絡先: (所属機関名) \newline%
		       (所属機関住所) \newline%
		       E-mail: }\quad%
	著者2氏名\afil{1,2}\\
	Author1\afil{1} \quad \quad Author2\afil{1,2}}

% 所属
\affiliation{
	\afil{1} 所属機関1 (和文)\\ 
	\afil{1} Affiliation 1 (English)\\
	\afil{2} 所属機関2\\
	\afil{2} Affiliation 2}

\abstract{
}

\maketitle
\thispagestyle{empty}

\section{はじめに}
aaa\cite{gSpan}
takigawa\cite{2016sys}


本章では、グラフ分類問題に対して
部分グラフ共起を用いたモデル学習手法を提案する。
また、実データを用いた実験により、
部分グラフ共起のもたらす効果を検証する。


\section{部分グラフ共起を用いた学習について}
%グラフ分類問題定義済み
%gBoost説明済み

本稿では、部分グラフの共起構造を考慮したモデル構築法を提案する。
まず、構築するモデルを定義する。
存在しうるすべての部分グラフ集合を$\mathcal{T}$とする。
その時、入力のグラフそれぞれに対して$|\mathcal{T}|$-次元のベクトル
\begin{equation*}
  \bm{x} = I(t \subseteq G_n) , \forall t \in \mathcal{T}
\end{equation*}
が考えられる。このベクトルに対して、
仮説(hypotheses)と呼ばれる、弱学習器を以下で定義する。
\begin{equation*}
  h(\bm{x};t,\omega) = \omega (2x_t - 1)
\end{equation*}
$\omega \in \Omega = \{-1,1\}$はパラメータである。
この時、以下のモデルを考える。
\begin{equation*}
  f(\bm{x}) = \sum_{(t,\omega) \in \mathcal{T}\times \Omega} \alpha_{t} h(\bm{x};t,\omega) + \sum \sum \alpha_{i,j} x_i x_j 
\end{equation*}
1つ目の項は1次の項であり、2つ目の項が部分グラフの共起を表現しており、$i$と$j$の部分グラフが同時に現れる場合を考えた2次の項である。
本研究では、3次以上の項は考えず、2つの部分グラフの共起のみ考慮する。

ここで、$|\mathcal{T}|$は、入力グラフ集合に対するすべての部分グラフ数存在し、
全列挙は現実的に困難であることが知られている。
これに対して共起を考えると、単純に$|\mathcal{T}|^2/2$の特徴量が増加する。
そのため、列挙をした後に学習する手法では解くことが現実ではないと考えられる。
よって、本稿では、既存手法であるgBoost法がブースティングの手法であることに着目して、
真に全列挙せず、必要なときに必要な部分のみを探索することで
全部分グラフとその共起を考慮したグラフ分類手法を提案する。
これを提案手法1:厳密法とし、\ref{st}節に示す。

提案手法1では、厳密に全部分グラフとその共起を用いたモデル構築が行う。
しかし、厳密に探索せずに良い特徴を選ぶことができれば、探索を減らし計算時間を削減できる可能性がある。
この時、適度に探索を少なくすることで
モデル構築にかかるコストを減らすことが期待できる。
そこで、提案手法2:Top-k法を提案し、\ref{tk}節に示す。

\subsection{提案手法1:厳密法}
\label{st}
全部分グラフとその共起を用いたグラフ分類の手法を提案する。
まずグラフ分類を、入力グラフに存在する部分グラフを特徴量としてモデル構築を考える。
この時、2種類のモデル学習手法が存在する。1つ目は、存在する部分グラフを列挙した後、
任意の学習モデルを構築する方法である。これは存在する部分グラフが膨大な数であるため、
一般的にグラフデータに対する出現数や部分グラフのサイズで探索を打ち切る必要がある。
2つ目は、探索を随時行いながらモデル構築を行う手法である。ブースティング法で
モデルを学習することで、毎回必要な部分グラフの探索を必要な部分のみ行うことで、
全部分グラフを用いたモデル学習を行うことができる手法である。

ここで、部分グラフの共起を考えた学習モデルを考える。
入力グラフに含まれる部分グラフを全列挙することは現実的ではないことは知られており、
先程述べたように、何らかの指標を用いて列挙をおこなったとする。
この時、列挙された部分グラフ数$N$に対して、その共起構造は$N^2/2$存在する。
部分グラフ数10000列挙したとすると、50000000の共起が存在する。
そのため、これらすべての特徴を用いた線形モデルの学習には、その数のパラメータの予測が必要である。
しかしこれに対して、一般的に用いられているデータ数は数百から数千であることが多い。
よって、実際に必要である特徴の数はそれほど多くないと考えられる。
今回は、L1正則化によって実際に使う特徴数を削減できる既存手法であるgBoost法をベースに
モデル構築を行うことで、必要な部分グラフやその共起を探索しつつ効率よくモデル構築を行う手法を提案する。

グラフ分類に対して、部分グラフとその共起を用いたモデルを構築するために、
gBoost法を拡張する。
基本的なアイデアは、gBoost法の繰り返し行われる部分グラフ探索
に共起を探索するように拡張するものである。
まず、グラフ探索を1度行い、この時最大gainを持つグラフが1つに決まる。
そして、共起を探索する際、boundがこの現在の最大gainを超えない部分グラフ
同士のみ共起を考えれば十分である。

\begin{algorithm}
  \small 
\caption{部分グラフとその共起の探索(厳密法)}\label{alg_cooc_st}
\begin{algorithmic}[1]
\Procedure{最適な部分グラフ、あるいはその共起}{}%\Comment{The g.c.d. of a and b}
\State グローバル変数 : $g^*,\omega^*,p^*,pc^*$
\State 最適な部分グラフ$g^*,\omega^*,p^*$を探索 \Comment{gBoost法の探索}
\ForAll{$p \in$1つのedgeからなるDFSコード}
\State project(p)
\EndFor
\EndProcedure

\Function{project}{p}
\If {pが最小DFSコードでない}
\State \textbf{return}
\EndIf
\State $p$に対するgain $g(p)$,bound $b(p)$を計算
\If {$b(p) < g^*$}
\State \textbf{return}
\EndIf
\ForAll{$t \in$1つのedgeからなるDFSコード}
\State Cooc\_project($p$,$t$)
\EndFor
\ForAll{$p^{\prime} \in$ Rightmost Extension of $p$}
\State project($p^{\prime}$)
\EndFor
\EndFunction

\Function{Cooc\_project}{$p,t$}
\If {$t$が最小DFSコードでない}
\State \textbf{return}
\EndIf
\If {$p<t$} \Comment{DFS辞書順}
\State \textbf{return}
\EndIf
\State $t$に対するgain $g(t)$,bound $b(t)$を計算
\If {$b(t) < g^*$}
\State \textbf{return}
\EndIf
\State $p,t$に対するgain $g(p,t)$,bound $b(p,t)$を計算
\If {$b(p,t) < g^*$}
\State \textbf{return}
\EndIf
\If {$\omega \in \Omega$それぞれに対して$g(p,t)> g^*$}
\State $g^* = g(p,t),p^* = p,\omega^* = \omega$
\EndIf
\ForAll{$t^{\prime} \in$ Rightmost Extension of $t$}
\State Cooc\_project($t^{\prime}$)
\EndFor
\EndFunction
\end{algorithmic}
\end{algorithm}

%アルゴリズムの説明
Algorithm\ref{alg_cooc_st}に擬似コードを示す。
まず、gBoost法と同様の部分グラフ探索を行い、
存在する部分グラフの中で、最大のgainを持つ部分グラフを発見する。
続いてproject関数を呼び出し、再帰的に共起を探索する1つ目の部分グラフを探索する。
この時、既に見つけている最適パターンを用いることで枝刈りが有効である。
そして、共起を考えられるパターンそれぞれに対してCooc\_project関数を適用する。
この時、1つ目と2つ目の部分グラフを入れ替えた実質同様の共起を探索しないために、
部分グラフ探索アルゴリズムであるgSpan\cite{gSpan}で用いられたグラフ表現であるDFSコード
とその辞書順(以下、DFS辞書順)を用いる。

アルゴリズムの効率化を図るために、
次に説明する2つの技巧を使うべきである。
このアルゴリズムにおいて、gainやboundの計算は
1回は計算量のそれほどかからない処理であるが、これが$|\mathcal{T}|^2/2$に
相当する数となると話は別である。
そのため、実際にアルゴリズム中で計算する共起の種類を可能な限り削減する必要がある。
まず、1つ目は、部分グラフ同士が包含関係にあるときである。
つまり、ある部分グラフが共起を考えようとしている部分グラフに含まれている場合、
その共起は片方の部分グラフと同じgain、boundを持つことがわかる。
そのため、そのような場合、gainやboundの計算は不要である。
しかし、これを判定するには部分グラフ同型判定問題を解く必要があるが、
この問題のクラスはNP完全であることがわかっている。
そのため、これを毎共起ごとに判定するのは非現実的である。
そこで、gSpanアルゴリズムにより各部分グラフはDFSコードで表されていることを用いて
一部効率化することを考える。ここでサイズ$l$の部分グラフがDFSコードで与えられた時、
そのDFSコードから1つずつ最後を除いたグラフがすべて存在し、共起の比較が存在する。
これに関してgainとboundの計算をせず現在のノードをスキップする事ができる。
2つ目は、最適パターンの性質に関する技巧である。
最適パターンは、親子関係にある部分グラフが同じ0-1のパターンを持つ時、
親を取るという性質がある。
よって、部分グラフを探索している時、親と同じ支持度を持つグラフに対して
共起を考える必要がない。
以上の2つを取り入れることで実際に共起を探索するノードを
削減することができる。



\subsection{提案手法2:Top-k法}
\label{tk}

提案手法1では厳密に全共起を探索することで、
全部分グラフとその共起すべてを用いたグラフ分類を行った。
しかし、計算時間の観点ですべてを見るのは非効率であることがある。
毎イテレーションで必要となる部分グラフが違うことに対して、
毎回すべての共起を見ず、適度に探索を効率化する手法が必要である。
そこで本稿では、共起を探索する際、ある指標を与えて共起を考慮する部分グラフ$k$個を決め、
その部分グラフと全部分グラフの共起を探索する手法を提案する。
このようにすることで、探索する特徴数は概算$k*|\mathcal{T}|$となり、
パラメータとして$k$を与え、$k=|\mathcal{T}|$の時、提案手法と同じモデル構築となるため、
パラメータkを与えることで、gBoost法と厳密法のトレードオフを達成する。
部分グラフパターンのgain $g(t)$、bound $b(t)$を用いて
\begin{equation}
  \lambda |g(t)| + (1 - \lambda) b(t) \label{lam}
\end{equation}
Top-kの指標をとする。
gainは直接的にその部分グラフの良さを表し、
boundはその部分グラフの子ノード、あるいはその部分グラフとの共起に
より良いパターンが見つかる可能性を表す。
そのため、gainとboundのトレードオフとなるこの指標を用いる。


\begin{algorithm}
  \small 
\caption{部分グラフとその共起の探索(Top-k)}\label{alg_cooc_topk}
\begin{algorithmic}[1]
\Procedure{最適な部分グラフ、あるいはその共起}{}%\Comment{The g.c.d. of a and b}
\State グローバル変数 : $g^*,\omega^*,p^*,pc^*$
\State 最適な部分グラフ$g^*,\omega^*,p^*$とTop-k部分グラフを探索 \Comment{gBoost法の探索}
\ForAll{$p \in $ Top-k 部分グラフ}
\State project(p)
\EndFor
\EndProcedure

\Function{project}{p}
\ForAll{$t \in$1つのedgeからなるDFSコード}
\State Cooc\_project($p$,$t$)
\EndFor
\EndFunction

\Function{Cooc\_project}{$p,t$}
\If {$t$が最小DFSコードでない}
\State \textbf{return}
\EndIf
\State $t$に対するgain $g(t)$,bound $b(t)$を計算
\If {$b(t) < g^*$}
\State \textbf{return}
\EndIf
\State $p,t$に対するgain $g(p,t)$,bound $b(p,t)$を計算
\If {$b(p,t) < g^*$}
\State \textbf{return}
\EndIf
\If {$\omega \in \Omega$それぞれに対して$g(p,t)> g^*$}
\State $g^* = g(p,t),p^* = p,\omega^* = \omega$
\EndIf
\ForAll{$t^{\prime} \in$ Rightmost Extension of $t$}
\State Cooc\_project($t^{\prime}$)
\EndFor
\EndFunction
\end{algorithmic}
\end{algorithm}

Algorithm\ref{alg_cooc_topk}に擬似コードを示す。
まず、厳密法と同様に、gBoost法と同様の部分グラフ探索を行い、
存在する部分グラフの中で、最大のgainを持つ部分グラフを発見する。
同時に、先に述べた指標(\ref{lam})にしたがって、Top-kの部分グラフを探索する。
そして、そのTop-kパターンそれぞれに対してCooc\_project関数を適用する。
この時、DFS辞書順による探索打ち切りはしない。

\subsection{効率的なモデル構築}

本節では、2つの提案手法に対して
より効率よくモデルを構築するための手法を提案する。
gBoost法は主問題をLPで定式化し、その双対問題を列生成法で解く。
この時、理論上追加する列の順に関係はない。
そこで、まず第一ステップとして、
既存手法であるgBoost法で通常の部分グラフで収束条件を満たすまで
部分グラフの追加を行う。
そしてその後、共起を探索するステップへと移行する。
このようにすることで、
本当に必要な共起を後に探索する構造となり、
探索する共起の数を削減することに期待できる。

これを各提案手法に対して適応した4種類の提案手法と
既存手法との関係について考察するために
\ref{exp_cooc}節で実データを用いた実験を行う。

\section{実験}
\label{exp_cooc}
\subsection{使用したデータセット}

\begin{table}
  \centering
\begin{tabular}{|c|r|r|r|r|r|}
\hline
データ名 & グラフ数 & 平均ノード数 & 平均エッジ数 & 正例 & 負例\\
\hline \hline
CPDB  & 684 & 25.2 & 25.6 & 341 & 343\\
\hline
Mutag & 188 & 26.3 & 28.1 & 125 & 63\\
\hline
CAS & 4337 & 30.3 & 31.3 & 2401 & 1936\\
\hline
AIDS(CAvsCM) & 1503 & 59.0 & 61.6 & 422 & 1081\\
\hline
\end{tabular}
\caption{使用したデータセット}
\label{dataset}
\end{table}

本稿では、Saigoら\cite{gBoost}によって用いられた4つのデータセットを使用した。
表\ref{dataset}に各データセットのグラフ数、平均ノード・エッジ数、正例と負例の数
を示す。

\subsection{精度に関する実験}
\subsubsection{実験設定}
\subsection{計算時間に関する実験}
\subsubsection{実験設定}

\section*{謝辞}

\bibliographystyle{jplain}
\bibliography{ref}


\end{document}

